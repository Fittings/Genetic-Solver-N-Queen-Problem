\documentclass[a4paper,11pt]{article}
\usepackage{a4wide}
%\usepackage{savetrees}
%\pagestyle{empty} % no page numbers
%\pagenumbering{gobble}  % really, no page numbers!

\title{COSC343 Assignment 2: n-Queens-Problem}
\author{Cameron Milsom}

\begin{document}

\maketitle

\section{Introduction}
To start off with, I decided to program it in python. This choice was purely because I was already learning ruby, but it wasn't specified in the assignment so I went with python instead.
In my assignment I came across a few issues. The main issue I had was due to poor lack of planning and foresight on my part.


\section{Planning}
I huge mistake that went into the assignment was lack of planning. In my head I had an idea of exactly what I wanted to do, but didn't put much thought into the consequences.
I originally spent a large amount of time designing a N-queens solution that focused on the positions of the individual queens rather than the position of all the queens on the board.
In my algorithm I was finding the perfect fitness of the queens position based on how many collisions it has with other queens. This was leading to a program that was finding queens in the same vertical position. This was the opposite of what I wanted.
The code I had written was vaguely correct and a long the right lines, however the general approach was a bit convulted (more than my current code) so I decided to re-approach the assignment from the beginning. 

\section*{Algorithms}

\subsection*{Selection}
I decided to go with Roulette Wheel selection. The roulette selection is designed so that it favours boards that have lower fitness values. The variation in fitness of the boards is not so significant that a particular board will be selected for the majority of the population. This makes it a decent fit for my genetic algorithm.

\subsection*{Crossover}
For my crossover function between parents I tried multiple functions to see if there would be any significant effect.
However, the program doesn't run at a speed fast enough to easily record a large quantity of values to find what algorithm is best.
To solve this, I tested these algorithms on a slightly lower n-value which can find solutions at a significantly faster rate.

\subsubsection*{Single Point Crossover}

\subsubsection*{Double Point Crossover}
\subsubsection*{Uniform Crossover}



\end{document}
